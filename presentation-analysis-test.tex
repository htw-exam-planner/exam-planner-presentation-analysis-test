\documentclass{beamer}
\usetheme{metropolis}

\usepackage[ngerman]{babel}

\title{SE II - Prüfungsplaner \\ Analyse und Test}
\author{Oliver von Seydlitz}
\date{Sommersemester 2018}

\begin{document}
  \maketitle

  \section{Analyse}
  \subsection{Aufgabe}
  \begin{frame}{\subsecname}
    \begin{itemize}
      \item Anforderungen ermitteln
      \item Ermittelte Anforderungen spezifizieren
      \item Geeignet beschreiben
      \item Verifizieren
    \end{itemize}
  \end{frame}

  \subsection{Angewandtes Wissen}
  \begin{frame}{\subsecname}
    Wissen aus Vorlesungen und Praktika SE I, insbesondere
    \begin{itemize}
      \item Was sind Anforderungen?
      \item Risiken der Anforderungsanalyse und deren Vermeidung
      \item Wie werden Anforderungen ermittelt?
      \item Wie werden Anforderungen beschrieben?
      \item Was muss ein Pflichtenheft enthalten?
    \end{itemize}
  \end{frame}

  \subsection{Herangehensweise}
  \begin{frame}{\subsecname}
    \begin{enumerate}
      \item Treffen am 23.4.:
      \begin{itemize}
        \item Definition Problem, Ziel
        \item Ermittlung nicht-funktionaler Anforderungen
      \end{itemize}
      \item Mögliche Anwendungsfälle definiert, dargestellt
      \item 1. Meilensteintreffen: Überarbeitung möglicher AWF, Klärung von Fragen
      \item Vollständige Spezifikation der AWF
      \item Zusammenstellung der Ergebnisse in Pflichtenheft
      \item 2. Meilensteintreffen: Vorstellung Pflichtenheft
      \item Überarbeitung Pflichtenheft
    \end{enumerate}
  \end{frame}

  \subsection{Erfahrungen}
  \begin{frame}{\subsecname}
    \begin{itemize}
      \item Wichtigkeit des Glossars
      \item Vollständigkeit nahezu unmöglich
      \item Inkonsistenzen treten schnell auf
        \begin{itemize}
          \item besonders, wenn mehrere am Pflichtenheft arbeiten
        \end{itemize}
      \item Herausforderung, fachlich statt technisch zu denken
      \item Vorsicht bei Verwendung von include- und extend-Beziehungen
      \item Pflichtenheft dient zur Kommunikation $\to$ Struktur wichtig
    \end{itemize}
  \end{frame}

  \subsection{Bewertung}
  \begin{frame}{\subsecname}
    \begin{block}{Was hat sich bewährt?}
      \begin{itemize}
        \item Vor dem Treffen mögliche AWF herausarbeiten
        \item Ein Glossar verwenden
        \item UML verwenden
        \item Oberflächenprototypen erstellen
      \end{itemize}
    \end{block}

    \begin{block}{Was würde ich anders machen}
      \begin{itemize}
        \item Mehr Treffen mit dem Kunden
        \item Nicht unbedingt UML und Satzschablonen
        \item Agiles Vorgehensmodell
      \end{itemize}
    \end{block}
  \end{frame}


  \section{Test}
  \subsection{Aufgabe}
  \begin{frame}{\subsecname}
    \begin{itemize}
      \item Ziel: Fehler finden
      \item Testfälle definieren
      \item Unit-Tests schreiben
      \item Tests durchführen
      \item Tests protokollieren
    \end{itemize}
  \end{frame}

  \subsection{Angewandtes Wissen}
  \begin{frame}{\subsecname}
    Wissen aus Software-Engineering I und II, insbesondere
    \begin{itemize}
      \item Was ist der Zusammenhang zwischen Analyse und Test?
      \item Wie werden Test dokumntiert?
      \item Welche Arten von Tests gibt es?
      \item Praktikum JUnit
    \end{itemize}
  \end{frame}

  \subsection{Herangehensweise}
  \begin{frame}{\subsecname}
    \begin{block}{Testgegenstand: Code und Dokumente}
      \begin{itemize}
        \item Alle Änderungen von Code und Dokumenten nur über Pull Request
        \item Jeder Pull Request von 2 Teammitgliedern geprüft
        \item Richtlinien: Qualitätsrichtlinien von D. Keiling
      \end{itemize}
    \end{block}
  \end{frame}

  \subsection{Herangehensweise}
  \begin{frame}{\subsecname}
    \begin{block}{Testgegenstand: Komponenten}
      \begin{itemize}
        \item Klassen des Models mit Unit Tests getestet
        \item Tests anhand Anforderungen im Pflichtenheft
      \end{itemize}
    \end{block}
  \end{frame}

  \subsection{Herangehensweise}
  \begin{frame}{\subsecname}
    \begin{block}{Testgegenstand: System}
      \begin{itemize}
        \item Systemtests für alle AWF definiert
        \item Alle wesentlichen Varianten
        \item Durchführung nach Fertigstellung jeder Ansicht
      \end{itemize}
    \end{block}
  \end{frame}

  \subsection{Erfahrungen}
  \begin{frame}{\subsecname}
    \begin{itemize}
      \item Viel mehr als nur Systemtests
      \item Bei hoher Komplexität Unit Tests unverzichtbar
    \end{itemize}
  \end{frame}

  \subsection{Bewertung}
  \begin{frame}{\subsecname}
      \begin{block}{Was hat sich bewährt?}
        \begin{itemize}
          \item Genauigkeit im Pflichtenheft $\to$ Vollständigkeit der Testfälle
          \item Automatisierte Tests
          \item Reviews auf GitHub
        \end{itemize}
      \end{block}

      \begin{block}{Was würde ich anders machen}
        \begin{itemize}
          \item Mehr automatisierte Tests
          \item Testgetriebene Entwicklung
        \end{itemize}
      \end{block}
  \end{frame}


\end{document}
